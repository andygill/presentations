\documentclass[smaller]{beamer}

\usepackage{eecs}

\usepackage{etex}
\usepackage[all]{xy}


%% ODER: format ==         = "\mathrel{==}"
%% ODER: format /=         = "\neq "
%
%
\makeatletter
\@ifundefined{lhs2tex.lhs2tex.sty.read}%
  {\@namedef{lhs2tex.lhs2tex.sty.read}{}%
   \newcommand\SkipToFmtEnd{}%
   \newcommand\EndFmtInput{}%
   \long\def\SkipToFmtEnd#1\EndFmtInput{}%
  }\SkipToFmtEnd

\newcommand\ReadOnlyOnce[1]{\@ifundefined{#1}{\@namedef{#1}{}}\SkipToFmtEnd}
\usepackage{amstext}
\usepackage{amssymb}
\usepackage{stmaryrd}
\DeclareFontFamily{OT1}{cmtex}{}
\DeclareFontShape{OT1}{cmtex}{m}{n}
  {<5><6><7><8>cmtex8
   <9>cmtex9
   <10><10.95><12><14.4><17.28><20.74><24.88>cmtex10}{}
\DeclareFontShape{OT1}{cmtex}{m}{it}
  {<-> ssub * cmtt/m/it}{}
\newcommand{\texfamily}{\fontfamily{cmtex}\selectfont}
\DeclareFontShape{OT1}{cmtt}{bx}{n}
  {<5><6><7><8>cmtt8
   <9>cmbtt9
   <10><10.95><12><14.4><17.28><20.74><24.88>cmbtt10}{}
\DeclareFontShape{OT1}{cmtex}{bx}{n}
  {<-> ssub * cmtt/bx/n}{}
\newcommand{\tex}[1]{\text{\texfamily#1}}	% NEU

\newcommand{\Sp}{\hskip.33334em\relax}


\newcommand{\Conid}[1]{\mathit{#1}}
\newcommand{\Varid}[1]{\mathit{#1}}
\newcommand{\anonymous}{\kern0.06em \vbox{\hrule\@width.5em}}
\newcommand{\plus}{\mathbin{+\!\!\!+}}
\newcommand{\bind}{\mathbin{>\!\!\!>\mkern-6.7mu=}}
\newcommand{\rbind}{\mathbin{=\mkern-6.7mu<\!\!\!<}}% suggested by Neil Mitchell
\newcommand{\sequ}{\mathbin{>\!\!\!>}}
\renewcommand{\leq}{\leqslant}
\renewcommand{\geq}{\geqslant}
\usepackage{polytable}

%mathindent has to be defined
\@ifundefined{mathindent}%
  {\newdimen\mathindent\mathindent\leftmargini}%
  {}%

\def\resethooks{%
  \global\let\SaveRestoreHook\empty
  \global\let\ColumnHook\empty}
\newcommand*{\savecolumns}[1][default]%
  {\g@addto@macro\SaveRestoreHook{\savecolumns[#1]}}
\newcommand*{\restorecolumns}[1][default]%
  {\g@addto@macro\SaveRestoreHook{\restorecolumns[#1]}}
\newcommand*{\aligncolumn}[2]%
  {\g@addto@macro\ColumnHook{\column{#1}{#2}}}

\resethooks

\newcommand{\onelinecommentchars}{\quad-{}- }
\newcommand{\commentbeginchars}{\enskip\{-}
\newcommand{\commentendchars}{-\}\enskip}

\newcommand{\visiblecomments}{%
  \let\onelinecomment=\onelinecommentchars
  \let\commentbegin=\commentbeginchars
  \let\commentend=\commentendchars}

\newcommand{\invisiblecomments}{%
  \let\onelinecomment=\empty
  \let\commentbegin=\empty
  \let\commentend=\empty}

\visiblecomments

\newlength{\blanklineskip}
\setlength{\blanklineskip}{1mm}

\newcommand{\hsindent}[1]{\quad}% default is fixed indentation
\let\hspre\empty
\let\hspost\empty
\newcommand{\NB}{\textbf{NB}}
\newcommand{\Todo}[1]{$\langle$\textbf{To do:}~#1$\rangle$}

\EndFmtInput
\makeatother
%


\newcommand{\hint}[2]
  {#1  \mbox{$\{$~{#2}~$\}$}}

\newcommand{\wrapunwrap}[3]
{$$\xymatrix@C=18pt@R=0pt{
        *[r]{\raisebox{-6pt}{#1\ ::\hspace*{-3pt}}}
      & *+[F-:]{#2} 
                \ar@/^2pc/[rr]^-{\ensuremath{\Varid{unwrap}}}
      & *[r]{\ensuremath{\Varid{work}}\ ::\ }
      & *+[F-:]{#3}="work"
                \ar@/^2pc/[ll]^-{\ensuremath{\Varid{wrap}}} 
}$$}

\newcommand{\wrapunwrapone}[3]
{$$\xymatrix@C=18pt@R=0pt{
        *[r]{\raisebox{-6pt}{#1\ ::\hspace*{-3pt}}}
      & *+[F-:]{#2} 
                \ar@`{(50,-30),(50,30)}_{Recursion}
      & *+[]{~}
      & *+[]{~}
}$$}

\newcommand{\wrapunwraptwo}[3]
{$$\xymatrix@C=18pt@R=0pt{
        *[r]{\raisebox{-6pt}{#1\ ::\hspace*{-3pt}}}
      & *+[F-:]{#2} 
                \ar@/^2pc/[rr]^-{\ensuremath{\Varid{unwrap}}}
                \ar@`{(30,-20),(30,20)}
      & *[r]{\ensuremath{\Varid{work}}\ ::\ }
      & *+[F-:]{#3}="work"
                \ar@/^2pc/[ll]^-{\ensuremath{\Varid{wrap}}} 
}$$}

\newcommand{\wrapunwrapthree}[3]
{$$\xymatrix@C=18pt@R=0pt{
        *[r]{\raisebox{-6pt}{#1\ ::\hspace*{-3pt}}}
      & *+[F-:]{#2} 
      & *[r]{\ensuremath{\Varid{work}}\ ::\ }
      & *+[F-:]{#3}="work"
                \ar@/^2pc/[ll]^-{\ensuremath{\Varid{wrap}}} 
                \ar@`{(85,-20),(85,20)}_{Recursion}
}$$}


\title{The Worker/Wrapper Transformation}
\author{Andy Gill\inst{1} \and Graham Hutton\inst{2}}
\institute{\inst{1}The University of Kansas \and \inst{2}The University of Nottingham}

\date{April 26th, 2012}

%\tinycolouredline{structure!25}% 
%{\color{white}\textbf{\insertshortauthor\hfill% 
%\insertshortinstitute}}% 
%\tinycolouredline{structure}% 
%{\color{white}\textbf{\insertshorttitle}\hfill}% 
%}}

%\logo{\includegraphics[width=1cm]{KUlogo1C}}

\begin{document}

\frame{\titlepage}

\begin{frame}
\frametitle{The Worker/Wrapper Transformation}


\begin{columns}[t] 
\begin{column}{0.8\textwidth}
\begin{block}{}
\begin{center}
The Worker/Wrapper Transformation is a rewrite technique\\
which changes the type of a (recursive) computation
\end{center}
\end{block}
\end{column}
\end{columns}


\vskip 0.1in

\begin{itemize}
\item Worker/wrapper has been used inside the Glasgow Haskell compiler since its inception to 
rewriting functions that use lifted values (thunks) into equivalent and more efficient functions that use unplifted values.
%\item No direct proof of correctness; syntactical arguments given
\item This talk will explain \alert{why} worker/wrapper works!
\item Much, much more general that just exploiting strictness analysis.
\item Worker/wrapper is about changing types
\end{itemize}
\end{frame}


\begin{frame}
\begin{columns}[t] 
\begin{column}{0.8\textwidth}
\begin{block}{Changing the type of a computation \ldots}
\begin{itemize} 
\item is pervasive in functional programming
\item useful in practice
\item the essence of turning a specification into a implementation
\end{itemize} 
\end{block}
\end{column}
\end{columns}
\vskip 0.2in
Thesis:

\begin{itemize} 
\item The worker/wrapper transformation is great technique for changing types
\end{itemize}
\vskip 0.2in

This talk
\begin{itemize}
\item Examples of \alert{what} Worker/Wrapper can do
% in longer talk
%\item Show how to use the worker/wrapper transformation
\item Formalize the Worker/Wrapper Transformation (\alert{why} it works)
\item Give a recipe for its use
\item Show \alert{how} to apply our worker/wrapper recipe to some examples
\end{itemize}
\end{frame}

\begin{frame}[fragile]
\frametitle{Example 1: Strictness Exploitation}

\begin{columns}[t] 
\begin{column}{0.8\textwidth}
\begin{block}{Before}
{\footnotesize\begin{semiverbatim}
\alert{fac :: Int -> Int -> Int}
fac n m = if n == 1 
          then m 
          else fac (n - 1) (m * n) 
\end{semiverbatim}}
\end{block}
\end{column}
\end{columns}

\vskip 0.2in
\begin{itemize}
\item \ensuremath{\Varid{n}} is trivially strict, \ensuremath{\Varid{m}} is provably strict
\item Can use \ensuremath{\Conid{Int}\mathbin{\#}}, a strict version of Int that is passed by value for \ensuremath{\Varid{n}} and \ensuremath{\Varid{m}}
\end{itemize}
\begin{columns}[t] 
\begin{column}{0.8\textwidth}
\begin{block}{After}
{\footnotesize\begin{semiverbatim}
fac n m = box (work (unbox n) (unbox m))

\alert{work :: Int# -> Int# -> Int#}
work n# m# = if n# ==# 1# 
            then m# 
            else work (n# -# 1#) (m# *# n#) 
\end{semiverbatim}}
\end{block}
\end{column}
\end{columns}

\end{frame}

\begin{frame}[fragile]
\frametitle{Example 2: Avoiding Needless Deconstruction}

\begin{columns}[t] 
\begin{column}{0.8\textwidth}
\begin{block}{Before}
{\footnotesize\begin{semiverbatim}
\alert{last       :: [a] -> a}
last []       = error "last: []"
last (x:[])   = x
last (x:xs)   = last xs
\end{semiverbatim}}
\end{block}
\end{column}
\end{columns}

\vskip 0.2in
\begin{itemize}
\item The recursive call of \ensuremath{\Varid{last}} never happens with an empty list
\item Subsequent recursive invocations performs a needless check for an empty list 
\end{itemize}
\begin{columns}[t] 
\begin{column}{0.8\textwidth}
\begin{block}{After}
{\footnotesize\begin{semiverbatim}
last []     = error "last: []"
last (x:xs) = work x xs

\alert{work :: a -> [a] -> a}
work x []     = x
work x (y:ys) = work y ys
\end{semiverbatim}}
\end{block}
\end{column}
\end{columns}

\end{frame}

\begin{frame}[fragile]
\frametitle{Example 3: Efficient \ensuremath{\Varid{nub}}}

\begin{columns}[t] 
\begin{column}{0.8\textwidth}
\begin{block}{Before}
{\footnotesize\begin{semiverbatim}
\alert{nub :: [Int] -> [Int]}
nub [] = []
nub (x:xs) = x : nub (filter (\\y -> not (x == y)) xs)
\end{semiverbatim}}
\end{block}
\end{column}
\end{columns}

\vskip 0.2in
{\footnotesize\begin{itemize}
\item \ensuremath{\Varid{filter}} is applied to the tail of the argument list on each recursive call, to avoid duplication
\item It would be more efficient to remember the elements that have already been issued
\end{itemize}}
\begin{columns}[t] 
\begin{column}{0.8\textwidth}
\begin{block}{After}
{\footnotesize\begin{semiverbatim}
nub :: [Int] -> [Int]
nub xs = work xs empty
 
\alert{work :: [Int] -> Set Int -> [Int]}
work xs except = 
  case dropWhile (\\ x -> x `member` except) xs of
    [] -> []
    (x:xs) -> x : work xs (insert x except)
\end{semiverbatim}}
\end{block}
\end{column}
\end{columns}

\end{frame}


\begin{frame}[fragile]
\frametitle{Example 4: Memoization}

\begin{columns}[t] 
\begin{column}{0.8\textwidth}
\begin{block}{Before}
{\footnotesize\begin{semiverbatim}
\alert{fib :: Nat -> Nat}
fib n = if n < 2 then 1 else fib (n-1) + fib (n-2)
\end{semiverbatim}}
\end{block}
\end{column}
\end{columns}

\vskip 0.2in
\begin{itemize}
\item Memoization is a well-known optimization for \ensuremath{\Varid{fib}}
\item Memoization is just a change in representation over the recursive call
\end{itemize}
\begin{columns}[t] 
\begin{column}{0.8\textwidth}
\begin{block}{After}
{\footnotesize\begin{semiverbatim}
fib :: Nat -> Nat
fib n = work !! n

\alert{work :: [Nat]}
work = map f [0..]
  where f = if n < 2 then 1 else work !! (n-1) + work !! (n-2)
\end{semiverbatim}}
\end{block}
\end{column}
\end{columns}

\end{frame}


\begin{frame}[fragile]
\frametitle{Example 5: Double-barrel CPS Translation}

\begin{columns}[t] 
\begin{column}{0.95\textwidth}
\begin{block}{Before}
{\scriptsize\begin{semiverbatim}
\alert{eval :: Expr -> Maybe Int}
eval (Val n)     =  Just n
eval (Add x y)   =  case eval x of Nothing -> Nothing
                                   Just n  -> case eval y of Nothing -> Nothing
                                                             Just m  -> Just (n+m)
eval (Throw)     =  Nothing
eval (Catch x y) =  case eval x of Nothing -> eval y
                                   Just n  -> Just n
\end{semiverbatim}}
\end{block}
\end{column}
\end{columns}
\vskip 0.1in
\begin{itemize}
\item CPS changes the result type from \ensuremath{\Conid{A}} to \ensuremath{(\Conid{A}\to \Conid{X})\to \Conid{X}}
\item Again, just a change in representation
\end{itemize}
\begin{columns}[t] 
\begin{column}{0.95\textwidth}
\begin{block}{After}

{\scriptsize\begin{semiverbatim}
eval e = work e Just Nothing

\alert{work :: Expr -> (Int -> Maybe Int) -> Maybe Int -> Maybe Int}
work (Val n)     s f  =  s n
work (Add x y)   s f  =  work x (\\n -> work y (\\m -> s (n+m)) f) f
work (Throw)     s f  =  f
work (Catch x y) s f  =  work x s (work y s f)
\end{semiverbatim}}
\end{block}
\end{column}
\end{columns}

\end{frame}

\begin{frame}
\begin{columns}[t] 
\begin{column}{0.8\textwidth}
\begin{block}{Changing the {\em representation\/} of a computation \ldots}
\begin{itemize} 
\item is pervasive in functional programming
\item useful in practice
\item the essence of turning a specification into a implementation
\item what worker/wrapper does
\end{itemize} 
\end{block}
\end{column}
\end{columns}
\end{frame}

\begin{frame}[fragile] 
\frametitle{Creating Workers and Wrappers for last}

\begin{semiverbatim}
last :: [a] -> a
\alert<4>{last} = \uncover<3->{\alert<3>{\\ v -> case v of
                []     -> error "last: []"
                (x:xs) -> \alert<4>{last_work} x xs}}

\uncover<2->{\alert<2>{last_work :: a -> [a] -> a
\alert<4>{last_work} = \\ x xs -> 
        (}}\\ v -> case v of
                  []     -> error "last: []"
                  (x:xs) -> case xs of
                             []    -> x
                             (_:_) -> \alert<4>{last} xs\uncover<2->{\alert<2>{) (x:xs)}}



\end{semiverbatim}
\begin{itemize}
\item <2- |alert@2>Create the worker out of the body and an invented coercion to the target type
\item <3- |alert@3>Invent the wrapper which call the worker
\item <4- |alert@4>These functions are mutually recursive
\end{itemize}
\end{frame}

\begin{frame}[fragile] 
\frametitle{Inline Wrapper}
\begin{semiverbatim}
last :: [a] -> a
last = \\ v -> case v of
                []     -> error "last: []"
                (x:xs) -> last_work x xs

last_work :: a -> [a] -> a
last_work = \\ x xs -> \only<1>{
        (\\ v -> case v of
                  []     -> error "last: []"
                  (x:xs) -> case xs of
                             []    -> x
                             (_:_) -> \alert{last} xs) (x:xs)



}\only<2>{
        (\\ v -> case v of
                  []     -> error "last: []"
                  (x:xs) -> case xs of
                             []    -> x
                             (_:_) -> 
               \alert{(\\ v -> case v of
                         []     -> error "last: []"
                         (x:xs) -> last_work x xs)} xs) (x:xs)
}\end{semiverbatim}
\begin{itemize}
\item<1- |alert@1>We now inline \ensuremath{\Varid{last}} inside \ensuremath{\Varid{last\char95 work}}
\item<2- |alert@2>\ensuremath{\Varid{last\char95 work}} is now trivially recursive.
\item<0>XXX
\end{itemize}
\end{frame}


\begin{frame}[fragile] 
\frametitle{Simplify \ensuremath{\Varid{work}}}
\begin{semiverbatim}
last :: [a] -> a
last = \\ v -> case v of
                []     -> error "last: []"
                (x:xs) -> last_work x xs

last_work :: a -> [a] -> a
\only<1>{\alert{last_work = \\ x xs -> 
       (\\ v -> case v of
                  []     -> error "last: []"
                  (x:xs) -> case xs of
                             []    -> x
                             (_:_) -> 
               (\\ v -> case v of
                         []     -> error "last: []"
                         (x:xs) -> last_work x xs) xs) (x:xs)
}}\only<2>{\alert{last_work = \\ x xs -> 
                \only<1>{\alert<1>{case (x:xs) of}
                  []     -> error "last: []"
                  \alert<1>{(x:xs)} -> }\only<2->{

                           }case xs of
                             []    -> x
                             (x:xs) -> last_work x xs



}}\end{semiverbatim}
\begin{itemize}
\item<1- |alert@1>We now simplify the worker
\item<2- |alert@2>Reaching our efficient implementation
\item<0>XXX
\end{itemize}
\end{frame}

\begin{frame}
\frametitle{The Informal Worker Wrapper Methodology}

\begin{itemize}
\item From a recursive function, construct two new functions
\end{itemize}
{\small\begin{columns}[t] 
\begin{column}{0.4\textwidth}
\begin{block}{Wrapper}
\begin{itemize} 
\item Replacing the original function
\item Coerces call to Worker
\end{itemize} 
\end{block}
\end{column} 
\begin{column}{0.4\textwidth}
\begin{block}{Worker}
\begin{itemize}
\item Performs main computation
\item Syntactically contains the body of the original function
\item Coerces call from Wrapper
\end{itemize} 
\end{block}
\end{column} 
\end{columns}}

\begin{itemize}
\item The initial worker and wrapper are mutually recursive
\item We then inline the wrapper inside the worker, and simplify
\item We end up with 
\begin{itemize}
\item An efficient recursive worker
\item An impedance matching non-recursive wrapper
\end{itemize}
\end{itemize}
\end{frame}

\begin{frame}{Questions about the Worker/Wrapper Transformation}
\begin{itemize}
\item Is the technique actually correct? 
\item How can this be proved? 
\item Under what conditions does it hold? 
\item How should it be used in practice? 
\end{itemize}

\end{frame}

\begin{frame}[fragile]
\frametitle{\ensuremath{\Varid{wrap}} and \ensuremath{\Varid{unwrap}}}

\wrapunwrap{last}{\ensuremath{[\mskip1.5mu \Varid{a}\mskip1.5mu]\to \Varid{a}}}{\ensuremath{\Varid{a}\to [\mskip1.5mu \Varid{a}\mskip1.5mu]\to \Varid{a}}}

\end{frame}

\begin{frame}[fragile]
\frametitle{\ensuremath{\Varid{wrap}} and \ensuremath{\Varid{unwrap}} in General}

\wrapunwrap{comp}{\ensuremath{\Conid{A}}}{\ensuremath{\Conid{B}}}

\end{frame}

\begin{frame}[fragile]
\begin{block}{Prerequisites}
\ensuremath{\Varid{comp}\mathbin{::}\Conid{A}}

\ensuremath{\Varid{comp}\mathrel{=}\Varid{fix}\;\Varid{body}} for some \ensuremath{\Varid{body}\mathbin{::}\Conid{A}\to \Conid{A}}

\vskip 0.1in

\ensuremath{\Varid{wrap}\mathbin{::}\Conid{B}\to \Conid{A}} is a coercion from type \ensuremath{\Conid{B}} to \ensuremath{\Conid{A}}

\ensuremath{\Varid{unwrap}\mathbin{::}\Conid{A}\to \Conid{B}} is a coercion from type \ensuremath{\Conid{A}} to \ensuremath{\Conid{B}}

\vskip 0.1in

\ensuremath{\Varid{wrap}\mathbin{\circ}\Varid{unwrap}} = $\ensuremath{\Varid{id}}_\ensuremath{\Conid{A}}$\hskip 1in{\em (basic worker/wrapper assumption)}
\end{block}

\only<1-5>{\begin{block}{Derivation}
\alert<1>{\ensuremath{\Varid{comp}\mathrel{=}\Varid{fix}\;\Varid{body}}}\\
\only<2->{\hint{$=$}{\ensuremath{\Varid{id}} is the identity for \ensuremath{\mathbin{\circ}}}\\
\alert<2>{\ensuremath{\Varid{comp}\mathrel{=}\Varid{fix}\;(\Varid{id}\mathbin{\circ}\Varid{body})}}\\}
\only<3->{\hint{$=$}{assuming \ensuremath{\Varid{wrap}\mathbin{\circ}\Varid{unwrap}\mathrel{=}\Varid{id}}}\\
\alert<3>{\ensuremath{\Varid{comp}\mathrel{=}\Varid{fix}\;(\Varid{wrap}\mathbin{\circ}\Varid{unwrap}\mathbin{\circ}\Varid{body})}}\\}
\only<4->{\hint{$=$}{rolling rule}\\
\alert<4>{\ensuremath{\Varid{comp}\mathrel{=}\Varid{wrap}\;(\Varid{fix}\;(\Varid{unwrap}\mathbin{\circ}\Varid{body}\mathbin{\circ}\Varid{wrap}))}}\\}
\only<5->{\hint{$=$}{define \ensuremath{\Varid{work}\mathrel{=}\Varid{fix}\;(\Varid{unwrap}\mathbin{\circ}\Varid{body}\mathbin{\circ}\Varid{wrap})}}\\
\alert<5>{\ensuremath{\Varid{comp}\mathrel{=}\Varid{wrap}\;\Varid{work}}\\
\ensuremath{\Varid{work}\mathrel{=}\Varid{fix}\;(\Varid{unwrap}\mathbin{\circ}\Varid{body}\mathbin{\circ}\Varid{wrap})}}\\}
\end{block}
}

\only<6>{\begin{block}{Worker/Wrapper Theorem}
If the above prerequisites hold, then
\[
\begin{parray}\SaveRestoreHook
\column{B}{@{}>{\hspre}l<{\hspost}@{}}%
\column{7}{@{}>{\hspre}l<{\hspost}@{}}%
\column{E}{@{}>{\hspre}l<{\hspost}@{}}%
\>[7]{}\Varid{comp}\mathrel{=}\Varid{fix}\;\Varid{body}{}\<[E]%
\ColumnHook
\end{parray}
\]\resethooks
can be rewritten as 
\[
\begin{parray}\SaveRestoreHook
\column{B}{@{}>{\hspre}l<{\hspost}@{}}%
\column{7}{@{}>{\hspre}l<{\hspost}@{}}%
\column{15}{@{}>{\hspre}l<{\hspost}@{}}%
\column{E}{@{}>{\hspre}l<{\hspost}@{}}%
\>[7]{}\Varid{comp}\mathrel{=}{}\<[15]%
\>[15]{}\Varid{wrap}\;\Varid{work}{}\<[E]%
\ColumnHook
\end{parray}
\]\resethooks
where \ensuremath{\Varid{work}\mathbin{::}\Conid{B}} is defined by
\[
\begin{parray}\SaveRestoreHook
\column{B}{@{}>{\hspre}l<{\hspost}@{}}%
\column{7}{@{}>{\hspre}l<{\hspost}@{}}%
\column{15}{@{}>{\hspre}l<{\hspost}@{}}%
\column{E}{@{}>{\hspre}l<{\hspost}@{}}%
\>[7]{}\Varid{work}\mathrel{=}{}\<[15]%
\>[15]{}\Varid{fix}\;(\Varid{unwrap}\mathbin{\circ}\Varid{body}\mathbin{\circ}\Varid{wrap}){}\<[E]%
\ColumnHook
\end{parray}
\]\resethooks
\end{block}}
\vskip 5in% to push things up
\end{frame}

\begin{frame}[fragile]
\frametitle{The worker/wrapper assumptions}
\begin{columns}[t] 
\begin{column}{0.1\textwidth}
\end{column}
\begin{column}{0.6\textwidth}
\begin{block}{Key step of proof}
\[
\begin{parray}\SaveRestoreHook
\column{B}{@{}>{\hspre}l<{\hspost}@{}}%
\column{E}{@{}>{\hspre}l<{\hspost}@{}}%
\>[B]{}\Varid{fix}\;(\Varid{id}\mathbin{\circ}\Varid{body}){}\<[E]%
\ColumnHook
\end{parray}
\]\resethooks
\hint{$=$}{assuming \ensuremath{\Varid{wrap}\mathbin{\circ}\Varid{unwrap}\mathrel{=}\Varid{id}}}
\[
\begin{parray}\SaveRestoreHook
\column{B}{@{}>{\hspre}l<{\hspost}@{}}%
\column{E}{@{}>{\hspre}l<{\hspost}@{}}%
\>[B]{}\Varid{fix}\;(\Varid{wrap}\mathbin{\circ}\Varid{unwrap}\mathbin{\circ}\Varid{body}){}\<[E]%
\ColumnHook
\end{parray}
\]\resethooks
\end{block}
\end{column}
\begin{column}{0.3\textwidth}
\end{column}

\end{columns}

\vskip 0.2in

We can actually use any of three different assumptions here

{\small
$$\begin{tabular}{rclr}
  \ensuremath{\Varid{wrap}\mathbin{\circ}\Varid{unwrap}} &= &\ensuremath{\Varid{id}} &(basic assumption)\\[5pt]
     &$\Downarrow$ \\[5pt]
  \ensuremath{\Varid{wrap}\mathbin{\circ}\Varid{unwrap}\mathbin{\circ}\Varid{body}} &= &\ensuremath{\Varid{body}} &(body assumption)\\[5pt]
     &$\Downarrow$ \\[5pt]
  \ensuremath{\Varid{fix}\;(\Varid{wrap}\mathbin{\circ}\Varid{unwrap}\mathbin{\circ}\Varid{body})} &= &\ensuremath{\Varid{fix}\;\Varid{body}} &(fix-point assumption)\\
  \end{tabular}
$$}
\end{frame}


\begin{frame}[fragile] 
\frametitle{The Worker/Wrapper Recipe}

\begin{block}{Recipe}
\begin{itemize}
\item Express the computation as a least fixed point;
\item Choose the desired new type for the computation;
\item Define conversions between the original and new types;
\item Check they satisfy one of the worker/wrapper assumptions;
\item Apply the worker/wrapper transformation;
\item Simplify the resulting definitions.
\end{itemize}
\end{block}

We simplify to remove the overhead of the \ensuremath{\Varid{wrap}} and \ensuremath{\Varid{unwrap}} coercions,
often using fusion, including the worker/wrapper fusion property.

\begin{block}{The Worker/Wrapper Fusion Property}
\begin{center}
If \ensuremath{\Varid{wrap}\mathbin{\circ}\Varid{unwrap}\mathrel{=}\Varid{id}}, then \ensuremath{(\Varid{unwrap}\mathbin{\circ}\Varid{wrap})\;\Varid{work}\mathrel{=}\Varid{work}}
\end{center}
\end{block}

\end{frame}

\begin{frame}[fragile] 
\frametitle{Creating Workers and Wrappers for last}

\wrapunwrap{last}{\ensuremath{[\mskip1.5mu \Varid{a}\mskip1.5mu]\to \Varid{a}}}{\ensuremath{\Varid{a}\to [\mskip1.5mu \Varid{a}\mskip1.5mu]\to \Varid{a}}}

\begin{semiverbatim}
wrap fn = \\ xs -> case xs of
                    [] -> error "last: []"
                    (x:xs) -> fn x xs

unwrap fn = \\ x xs -> fn (x:xs)

last = fix body

body last = \\ v -> case v of
                     []     -> error "last: []"
                     (x:[]) -> x
                     (x:xs) -> last xs
\end{semiverbatim}
\end{frame}

\begin{frame}[fragile] 
\frametitle{Testing the basic worker/wrapper assumption:\\Does \ensuremath{\Varid{wrap}\mathbin{\circ}\Varid{unwrap}\mathrel{=}\Varid{id}}?}
\begin{semiverbatim}
wrap . unwrap 
\end{semiverbatim}
\hint{$=$}{apply \ensuremath{\Varid{wrap}}, \ensuremath{\Varid{unwrap}} and \ensuremath{\mathbin{\circ}}}
\begin{semiverbatim}
\\ fn ->
  (\\ xs -> case xs of
              [] -> error "last: []"
              (x:xs) -> (\\ x xs -> fn (x:xs)) x xs)
\end{semiverbatim}
\hint{$=$}{$\beta$-reduction}
\begin{semiverbatim}
\\ fn ->
  (\\ xs -> case xs of
              [] -> error "last: []"
              (x:xs) -> fn (x:xs))
\end{semiverbatim}

Clearly not equal to \ensuremath{\Varid{id}\mathbin{::}([\mskip1.5mu \Varid{a}\mskip1.5mu]\to \Varid{a})\to ([\mskip1.5mu \Varid{a}\mskip1.5mu]\to \Varid{a})}

\end{frame}

\begin{frame}[fragile] 
\frametitle{Testing the body worker/wrapper assumption:\\Does \ensuremath{\Varid{wrap}\mathbin{\circ}\Varid{unwrap}\mathbin{\circ}\Varid{body}\mathrel{=}\Varid{body}}?}
\begin{semiverbatim}
wrap . unwrap . body
\end{semiverbatim}
\hint{$=$}{apply \ensuremath{\Varid{wrap}}, \ensuremath{\Varid{unwrap}} and \ensuremath{\mathbin{\circ}}}
\only<2->{\\\hint{$=$}{$\beta$-reductions}}%
\only<3->{\\\hint{$=$}{case of known constructors}}%
\only<4->{\\\hint{$=$}{common up case}}%
\begin{semiverbatim}
\only<1>{(\\ fn ->
  (\\ xs -> case xs of
              [] -> error "last: []"
              (x:xs) -> (\\ x xs -> fn (x:xs)) x xs)) 
   (\\ last v -> case v of
                     []     -> error "last: []"
                     (x:[]) -> x
                     (x:xs) -> last xs)
}\only<2>{(\\ fn ->
  (\\ xs -> case xs of
              [] -> error "last: []"
              (x:xs) -> case (x:xs) of
                     []     -> error "last: []"
                     (x:[]) -> x
                     (x:xs) -> fn xs))
}\only<3>{(\\ fn ->
  (\\ xs -> case xs of
              [] -> error "last: []"
              (x:xs) -> case xs of
                          [] -> x
                          xs -> fn xs))
}\only<4>{(\\ fn ->
  (\\ xs -> case xs of
              [] -> error "last: []"
              (x:[]) -> x
              (x:xs) -> fn xs))
}\end{semiverbatim}

\only<4->{Which equals \ensuremath{\Varid{body}}. QED.}
\end{frame}

\begin{frame}[fragile] 
\frametitle{Applying the Worker/Wrapper Transformation}
\begin{columns}[t] 
\begin{column}{0.3\textwidth}
\begin{block}{Before}\[
\begin{parray}\SaveRestoreHook
\column{B}{@{}>{\hspre}l<{\hspost}@{}}%
\column{3}{@{}>{\hspre}l<{\hspost}@{}}%
\column{E}{@{}>{\hspre}l<{\hspost}@{}}%
\>[3]{}\Varid{last}\mathrel{=}\Varid{fix}\;\Varid{body}{}\<[E]%
\ColumnHook
\end{parray}
\]\resethooks
\end{block}
\end{column}
\begin{column}{0.7\textwidth}
\wrapunwrap{last}{\ensuremath{[\mskip1.5mu \Varid{a}\mskip1.5mu]\to \Varid{a}}}{\ensuremath{\Varid{a}\to [\mskip1.5mu \Varid{a}\mskip1.5mu]\to \Varid{a}}}
\end{column}
\end{columns}

\begin{semiverbatim}
last  :: [a] -> a
last = wrap work

work :: a -> [a] -> a
work = fix ( unwrap 
           . body 
           . wrap
           )
\end{semiverbatim}
\end{frame}


\begin{frame}[fragile] 
\frametitle{Inline and Simplify}
\begin{semiverbatim}
last  :: [a] -> a
last xs = case xs of
           [] -> error "last: []"
           (x:xs) -> work x xs

work :: a -> [a] -> a
\only<1>{work = fix ( (\\ fn x xs -> fn (x:xs))
           . (\\ last v -> case v of
                     []     -> error "last: []"
                     (x:[]) -> x
                     (x:xs) -> last xs)
           . (\\ fn xs -> case xs of
                    [] -> error "last: []"
                    (x:xs) -> fn x xs)
           )
}\only<2>{work = fix ( \\ fn x xs ->
                  case (x:xs) of
                    []     -> error "last: []"
                    (x:[]) -> x
                    (x:xs) -> case xs of
                                [] -> error "last: []"
                                (x:xs) -> fn x xs
           )
}\only<3>{work = fix ( \\ fn x xs ->
                  case xs of
                    [] -> x
                    xs -> case xs of
                           [] -> error "last: []"
                           (x:xs) -> fn x xs
           )
}\only<4>{work = fix ( \\ fn x xs ->
                  case xs of
                    [] -> x
                    (x:xs) -> fn x xs
           )
}\only<5>{work x xs = case xs of
             []     -> x
             (x:xs) -> work x xs
}\end{semiverbatim}
\end{frame}


\begin{frame}[fragile] 
\frametitle{When does the Worker/Wrapper Transformation succeed?}

When \ensuremath{\Varid{unwrap}\mathbin{\circ}\Varid{wrap}} fuse! 

\wrapunwrap{comp}{\ensuremath{\Conid{A}}}{\ensuremath{\Conid{B}}}

Emerging heuristic

\begin{columns}[t] 
\begin{column}{0.42\textwidth}
\begin{block}{Worker/Wrapper Fusion}
\begin{center}
Pre-condition: \ensuremath{\Varid{wrap}\mathbin{\circ}\Varid{unwrap}\mathrel{=}\Varid{id}}$_{\ensuremath{\Conid{A}}}$\\
~\\[5pt]
When B is ``larger'' than A
\end{center}
\end{block}
\end{column}

\begin{column}{0.04\textwidth}
\end{column}

\begin{column}{0.42\textwidth}
\begin{block}{Evaluation}
\begin{center}
Pre-condition: \ensuremath{\Varid{unwrap}\mathbin{\circ}\Varid{wrap}\mathrel{=}\Varid{id}}$_{\ensuremath{\Conid{B}}}$\\
~\\[5pt]
When A is ``larger'' than B
\end{center}
\end{block}
\end{column}
\end{columns}

{~}\\[5pt]
More powerful fusion methods can also be used

\end{frame}


\begin{frame}
\frametitle{Conclusions}

\begin{itemize}
\item Worker/wrapper is a general and systematic approach to transforming
a computation of one type into an equivalent computation of another type

\item It is straightforward to understand and apply, requiring
only basic equational reasoning techniques, and often avoiding
the need for induction

\item It allows many seemingly unrelated optimization
techniques to be captured inside a single unified framework
\end{itemize}

\end{frame}

%\begin{frame}
%\frametitle{Further Work}
%
%\begin{itemize}
%\item Monadic and Effectful Constructions
%\item Mechanization
%\item Implement inside the Haskell Equational Reasoning Assistant
%\item Consider other patterns of recursion
%\end{itemize}
%
%\end{frame}

\end{document}
